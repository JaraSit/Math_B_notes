\documentclass{article}

% Language setting
% Replace `english' with e.g. `spanish' to change the document language
\usepackage[english]{babel}

% Set page size and margins
% Replace `letterpaper' with `a4paper' for UK/EU standard size
\usepackage[letterpaper,top=2cm,bottom=2cm,left=3cm,right=3cm,marginparwidth=1.75cm]{geometry}

% Useful packages
\usepackage{amsmath}
\usepackage{graphicx}
\usepackage[colorlinks=true, allcolors=blue]{hyperref}
\usepackage{empheq}
\usepackage{amsthm}
\usepackage{bm}
\usepackage{color}
\usepackage{xcolor}

\theoremstyle{definition}
\setcounter{section}{8}
\newtheorem{exmp}{Příklad}[section]

\newcommand{\e}[1]{\mathrm{e}^{#1}}
\newcommand{\vect}[1]{\bm{#1}}

\newenvironment{boxx}
    {\begin{center}
    \begin{tabular}{|p{0.9\textwidth}|}
    \hline\\
    }
    { 
    \\\\\hline
    \end{tabular} 
    \end{center}
    }

\newsavebox{\selvestebox}
\newenvironment{colbox}[1]
  {\newcommand\colboxcolor{#1}%
   \begin{lrbox}{\selvestebox}%
   \begin{minipage}{\dimexpr\columnwidth-2\fboxsep\relax}}
  {\end{minipage}\end{lrbox}%
   \begin{center}
   \colorbox[HTML]{\colboxcolor}{\usebox{\selvestebox}}
   \end{center}}

\graphicspath{{figs/}}

\title{Matematika B - Cvičení 8}
\date{}
%\author{You}

\begin{document}
\maketitle

%%%%%%%%%%%%%%%%%%%%%%%%%%%%%%%%%%%%%%%%%%%%%%%%%%%%%
%%%%%%%%%%%%%%%%%%%%%%%%%%%%%%%%%%%%%%%%%%%%%%%%%%%%%
%% 08. CVICENI
%%%%%%%%%%%%%%%%%%%%%%%%%%%%%%%%%%%%%%%%%%%%%%%%%%%%%
%%%%%%%%%%%%%%%%%%%%%%%%%%%%%%%%%%%%%%%%%%%%%%%%%%%%%
%\clearpage
%\newpage
%\section{Cvičení 8: Lokální extrémy funkcí}
%Lokální extrémy funkce 2 proměnných (stacionární body, Hessova matice). Metoda nejmenších čtverců.

%%%%%%%%%%%%%%%%%%%%%%%%%%
%%%%% TEORIE
%%%%%%%%%%%%%%%%%%%%%%%%%%
\noindent\fbox{%
    \parbox{\textwidth}{%
        \framebox[1.0\columnwidth]{Extremálnost bodu}\nonumber
        \vspace{2mm}
        Pokud pro danou funkci $f(x,y)$ platí v bodě $(x_0, y_0)$ následující podmínka
        \begin{align}
            \frac{\partial f}{\partial x}(x_0, y_0)=0 \quad \land \quad \frac{\partial f}{\partial y}(x_0, y_0)=0,
        \end{align}
        potom bod $(x_0, y_0)$ nazýváme \textbf{stacionárním bodem} funkce $f(x,y)$. Další rozhodování o extremálnosti leží ve vyčíslení Hessiánu $H_f(x,y)$, který je definovám jako determinant Hessovy matice
        \begin{align}
            \begin{pmatrix}
                \frac{\partial^2 f}{\partial x^2}(x, y) & \frac{\partial^2 f}{\partial x\partial y}(x, y) \\
                \frac{\partial^2 f}{\partial y\partial x}(x, y) &
                \frac{\partial^2 f}{\partial y^2}(x, y).
            \end{pmatrix}
        \end{align}
        Tedy Hessián lze vyjádřit jako
        \begin{align}
            H_f(x,y)=\frac{\partial^2 f}{\partial x^2}\frac{\partial^2 f}{\partial y^2}-\left(\frac{\partial^2 f}{\partial x\partial y}\right)^2.
        \end{align}
        Potom platí:
        \begin{enumerate}
            \item Je-li $H_f(x_0, y_0)>0$ má funkce $f$ v bodě $(x_0, y_0)$\ potom
                \begin{enumerate}
                    \item $\frac{\partial^2 f}{\partial x^2}(x_0, y_0)>0$ má funkce v bodě $(x_0, y_0)$ ostré lokální minimum,
                    \item $\frac{\partial^2 f}{\partial x^2}(x_0, y_0)<0$ má funkce v bodě $(x_0, y_0)$ ostré lokální maximum.
                \end{enumerate}
            \item Je-li $H_f(x_0, y_0)<0$ má funkce $f$ v bodě $(x_0, y_0)$ sedlový bod
            \item Je-li $H_f(x_0, y_0)=0$ nevíme zda má $f$ v bodě $(x_0, y_0)$ lokální extrém
        \end{enumerate}
    }%
}

%%%%%%%%%%%%%%%%%%%%%%%%%%
%%%%% PŘÍKLAD
%%%%%%%%%%%%%%%%%%%%%%%%%%
\begin{colbox}{DDDDDD}
\begin{exmp}
Určete lokální extrémy funkce $f(x,y)=xy\e{x-y^2/2}$.
\end{exmp}
\end{colbox}
Podezřelé body: $(x_0, y_0)=(0, 0); (x_1, y_1)=(-1, 1); (x_2, y_2)=(-1, -1)$
\begin{center}
%\centering
\begin{tabular}{cccc}
    \centering
    Podezřelý bod & Hessova matice & Hessián & Klasifikace \\
    \hline
    Bod $(0, 0)$ & $\bar{H}=
    \begin{pmatrix}
        0 & 1 \\
        1 & 0
    \end{pmatrix} $ & $H_f=-1$ & Sedlo \\
    Bod $(-1, 1)$ & $\bar{H}=
    \begin{pmatrix}
        1 & 0 \\
        0 & 2
    \end{pmatrix} $ & $H_f=2$ & Lokální minimum \\
    Bod $(-1, -1)$ & $\bar{H}=
    \begin{pmatrix}
        -1 & 0 \\
        0 & -2
    \end{pmatrix} $ & $H_f=2$ & Lokální maximum \\
\end{tabular}
\end{center}

%%%%%%%%%%%%%%%%%%%%%%%%%%
%%%%% PŘÍKLAD
%%%%%%%%%%%%%%%%%%%%%%%%%%
\begin{colbox}{DDDDDD}
\begin{exmp}
Určete lokální extrémy funkce $f(x,y)=xy\e{x-y^2/2}$.
\end{exmp}
\end{colbox}

%%%%%%%%%%%%%%%%%%%%%%%%%%
%%%%% PŘÍKLAD
%%%%%%%%%%%%%%%%%%%%%%%%%%
\begin{colbox}{ADD8E6}
\textbf{Cvičení:}
\end{colbox}

%%%%%%%%%%%%%%%%%%%%%%%%%%
%%%%% TEORIE
%%%%%%%%%%%%%%%%%%%%%%%%%%
\noindent\fbox{%
    \parbox{\textwidth}{%
        \framebox[1.0\columnwidth]{Metoda nejmenších čtverců}\nonumber
        \vspace{2mm}
        Pokud dvojicí získaných dat $\{x_i, y_i\}_{i=1}^n$ chci proložit co nejlépe lineární křivku $y=ax+b$, potom lze parametry $a, b$ najít optimalizační úlohou. Uvažujme nejprve, že celková chyba $e(a,b)$ je dána jako čtverec rozdílu mezi změřenými a odhadovanými daty, tedy
        \begin{align*}
            e(a,b)=\sum_{i=1}^n\left(ax_i + b - y_i\right)^2,
        \end{align*}
        potom hledám nejlepší parametry $a^*, b^*$ takové, že minimalizují celkovou chybu. Tedy pro tyto parametry musí platit:
        \begin{align*}
            \frac{\partial e}{\partial a}(a^*, b^*)=0,\quad\land\quad \frac{\partial e}{\partial b}(a^*, b^*)=0,
        \end{align*}
        Tedy
        \begin{align*}
            2\sum_{i=1}^n\left(ax_i + b - y_i\right)x_i=0, \\
            2\sum_{i=1}^n\left(ax_i + b - y_i\right)=0.
        \end{align*}
        A po úpravě konečně dostáváme soustavu
        \begin{align*}
            a\sum_{i=1}^n x_i^2 + b\sum_{i=1}^n x_i&=\sum_{i=1}^n x_iy_i, \\
            a\sum_{i=1}^n x_i + bn&=\sum_{i=1}^n y_i\\
        \end{align*}
        
    }%
}

%%%%%%%%%%%%%%%%%%%%%%%%%%
%%%%% PŘÍKLAD
%%%%%%%%%%%%%%%%%%%%%%%%%%
\begin{colbox}{DDDDDD}
\begin{exmp}
Metodou nejmenších čtverců aproximujte data křivkou $y=ax+b$.

\begin{center}
\begin{tabular}{|c|c|c|c|c|c|}
    \hline
    $x_i$ & 0 & 1 & 3 & 5 & 6 \\
    \hline
    $y_i$ & 5 & 3 & 3 & 2 & 1 \\
    \hline
\end{tabular}
\end{center}

\end{exmp}
\end{colbox}

Řešení: $y=-0.538x + 4.415$


\end{document}