\documentclass{article}

% Language setting
% Replace `english' with e.g. `spanish' to change the document language
\usepackage[english]{babel}

% Set page size and margins
% Replace `letterpaper' with `a4paper' for UK/EU standard size
\usepackage[letterpaper,top=2cm,bottom=2cm,left=3cm,right=3cm,marginparwidth=1.75cm]{geometry}

% Useful packages
\usepackage{amsmath}
\usepackage{graphicx}
\usepackage[colorlinks=true, allcolors=blue]{hyperref}
\usepackage{empheq}
\usepackage{amsthm}
\usepackage{bm}
\usepackage{color}
\usepackage{xcolor}

\theoremstyle{definition}
\newtheorem{exmp}{Příklad}[section]
\setcounter{section}{11}

\newcommand{\e}[1]{\mathrm{e}^{#1}}
\newcommand{\vect}[1]{\bm{#1}}

\newenvironment{boxx}
    {\begin{center}
    \begin{tabular}{|p{0.9\textwidth}|}
    \hline\\
    }
    { 
    \\\\\hline
    \end{tabular} 
    \end{center}
    }

\newsavebox{\selvestebox}
\newenvironment{colbox}[1]
  {\newcommand\colboxcolor{#1}%
   \begin{lrbox}{\selvestebox}%
   \begin{minipage}{\dimexpr\columnwidth-2\fboxsep\relax}}
  {\end{minipage}\end{lrbox}%
   \begin{center}
   \colorbox[HTML]{\colboxcolor}{\usebox{\selvestebox}}
   \end{center}}

\graphicspath{{figs/}}

\title{Matematika B - Cvičení 11}
\date{}
%\author{You}

\begin{document}
\maketitle

%%%%%%%%%%%%%%%%%%%%%%%%%%%%%%%%%%%%%%%%%%%%%%%%%%%%%
%%%%%%%%%%%%%%%%%%%%%%%%%%%%%%%%%%%%%%%%%%%%%%%%%%%%%
%% 11. CVICENI
%%%%%%%%%%%%%%%%%%%%%%%%%%%%%%%%%%%%%%%%%%%%%%%%%%%%%
%%%%%%%%%%%%%%%%%%%%%%%%%%%%%%%%%%%%%%%%%%%%%%%%%%%%%
%\clearpage
%\newpage
%\section{Cvičení 9: Implicitně definované funkce}

%%%%%%%%%%%%%%%%%%%%%%%%%%
%%%%% TEORIE
%%%%%%%%%%%%%%%%%%%%%%%%%%
\noindent\fbox{%
    \parbox{\textwidth}{%
        \framebox[1.0\columnwidth]{Existence implicitně zadané funkce}\nonumber
        \vspace{2mm}
        Vektorové pole $F$ se nazývá potenciální, pokud postupně všechny jeho složky lze získat příslušnou derivací určité funkce $U$, kterou nazýváme potenciálem. Např. pro dvourozměrný případ musí platit
        \begin{align*}
            F_1=\frac{\partial U}{\partial x}, \quad F_2=\frac{\partial U}{\partial y}, \quad \text{pro } F=(F_1, F_2)
        \end{align*}
        Podobně pro tří dimensionální funkci $F$ platí, že je potenciální pokud
        \begin{align*}
            F_1=\frac{\partial U}{\partial x}, \quad F_2=\frac{\partial U}{\partial y}, \quad
            F_3=\frac{\partial U}{\partial z}, \quad \text{pro } F=(F_1, F_2, F_3)
        \end{align*}
    }%
}

%%%%%%%%%%%%%%%%%%%%%%%%%%
%%%%% PŘÍKLAD
%%%%%%%%%%%%%%%%%%%%%%%%%%
\begin{colbox}{DDDDDD}
\begin{exmp}
Ověřte, že funkce $U(x,y)=x^2y + \frac{y^3}{3}$ je potenciálem vektorového pole $F(x,y)=(2xy, x^2+y^2)$
\end{exmp}
\end{colbox}
Řešení: Ano

%%%%%%%%%%%%%%%%%%%%%%%%%%
%%%%% TEORIE
%%%%%%%%%%%%%%%%%%%%%%%%%%
\noindent\fbox{%
    \parbox{\textwidth}{%
        \framebox[1.0\columnwidth]{Nezávislost na integrační cestě}\nonumber
        \vspace{2mm}
        $\int_\mathcal{K}F\cdot\mathrm{d}r$ nezávisí na integrační cestě, pouze na poloze počátečního a koncového bodu cesty, právě tehdy, když je $F$ potenciální.
    }%
}

%%%%%%%%%%%%%%%%%%%%%%%%%%
%%%%% TEORIE
%%%%%%%%%%%%%%%%%%%%%%%%%%
\noindent\fbox{%
    \parbox{\textwidth}{%
        \framebox[1.0\columnwidth]{Potenciální pole}\nonumber
        \vspace{2mm}
            Jestliže platí 
            \begin{align*}
                \frac{\partial F_1}{\partial y} = \frac{\partial F_2}{\partial x}, \quad \text{pro } n=2 \\ 
                \frac{\partial F_1}{\partial y} = \frac{\partial F_2}{\partial x}, \frac{\partial F_1}{\partial z} = \frac{\partial F_3}{\partial x}, \frac{\partial F_2}{\partial z} = \frac{\partial F_3}{\partial y}, \quad \text{pro } n=3,
            \end{align*}
            potom je $F$ potenciální.
    }%
}


\end{document}