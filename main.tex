\documentclass{article}

% Language setting
% Replace `english' with e.g. `spanish' to change the document language
\usepackage[english]{babel}

% Set page size and margins
% Replace `letterpaper' with `a4paper' for UK/EU standard size
\usepackage[letterpaper,top=2cm,bottom=2cm,left=3cm,right=3cm,marginparwidth=1.75cm]{geometry}

% Useful packages
\usepackage{amsmath}
\usepackage{graphicx}
\usepackage[colorlinks=true, allcolors=blue]{hyperref}
\usepackage{empheq}
\usepackage{amsthm}
\usepackage{bm}
\usepackage{color}
\usepackage{xcolor}

\theoremstyle{definition}
\newtheorem{exmp}{Příklad}[section]

\newcommand{\e}[1]{\mathrm{e}^{#1}}
\newcommand{\vect}[1]{\bm{#1}}

\newenvironment{boxx}
    {\begin{center}
    \begin{tabular}{|p{0.9\textwidth}|}
    \hline\\
    }
    { 
    \\\\\hline
    \end{tabular} 
    \end{center}
    }

\newsavebox{\selvestebox}
\newenvironment{colbox}[1]
  {\newcommand\colboxcolor{#1}%
   \begin{lrbox}{\selvestebox}%
   \begin{minipage}{\dimexpr\columnwidth-2\fboxsep\relax}}
  {\end{minipage}\end{lrbox}%
   \begin{center}
   \colorbox[HTML]{\colboxcolor}{\usebox{\selvestebox}}
   \end{center}}

\graphicspath{{figs/}}

\title{Matematika B}
%\author{You}

\begin{document}
\maketitle

%%%%%%%%%%%%%%%%%%%%
%% SYLABUS
%%%%%%%%%%%%%%%%%%%%
\section*{Sylabus}
\begin{enumerate}
    \item Vektory a matice, operace s vektory, lineární kombinace vektorů, lineární závislost a nezávislost, operace s maticemi, hodnost matice (i s parametrem). Soustavy lin. algebraických rovnic. Frobeniova věta.
    \item Soustavy lin. algebraických rovnic s parametrem. Determinanty, inverzní matice, výpočet inverzní matice, maticové rovnice. Vlastní čísla matic, určování vl. čísel a vl. vektorů matice - stačí 2x2. Neučí se Cramerovo pravidlo.
    \item Stručně geometrie v Rn (zejména v R2 a v R3). Metrika a norma vektoru v Euklidovském prostoru, význam skalárního a vektorového součinu. Parametrické rovnice přímky a roviny, obecná rovnice nadroviny. Geometrický význam soustav lin. algebraických rovnic. Příklady na vzájemnou polohu, vzdálenosti. Neučí se úhel vektorů a tedy ani odchylky přímek a rovin - pouze kolmost.
    \item Vlastnosti množin v Euklidovském prostoru (otevřená, uzavřená, omezená konvexní, obloukově souvislá). Funkce n reálných proměnných a její definiční obor.
    \item Graf funkce 2 proměnných, vrstevnice. Spojitost a limita jen velmi okrajově (na cvičeních lze i zcela vynechat). Zobrazení z Rn do Rk.
    \item Parciální derivace funkce více proměnných, gradient, derivace ve směru, derivace zobrazení (=Jakobiho matice). Derivování složených funkcí.
    \item Aplikace derivací funkce 2 proměnných (tečná rovina, totální diferenciál, Taylorův polynom, Newtonova metoda pro soustavy 2 rovnic).
    \item Lokální extrémy funkce 2 proměnných (stacionární body, Hessova matice). Metoda nejmenších čtverců.
    \item Implicitně zadané funkce (1 a 2 proměnných), derivace implicitně zadané funkce.
    \item Křivky zadané parametricky v R2 a v R3, orientace křivky. Vektorové pole v R2 a v R3. Křivkový integrálvektorového pole, práce síly. Neučí se křivkový integrál skalárního pole.
    \item Nezávislost křivkového integrálu na integrační cestě. Potenciál vektorového pole. Určování potenciálu v R2 a v R3. Neučí se pojem jednoduše souvislá oblast, místo ní uvádíme slabší větu pro konvexní oblast.
    \item Dvojný integrál a jeho geometrický význam. Výpočet dvojného integrálu postupnou integrací - Fubiniova věta. Polární souřadnice. Substituce pro dvojný integrál (stačí jen do polárních souřadnic. Neučí se trojný integrál.
    \item Soustavy dvou diferenciálních rovnic 1.řádu. Řešení autonomních soustav lineárních diferenciálních rovnic s konstantními koeficienty. Stačí jen pro 2 různá reálná či komplexní vlastní čísla, případ dvojnásobného vlastního čísla se neučí.
\end{enumerate}

%%%%%%%%%%%%%%%%%%%%%%%%%%%%%%%%%%%%%%%%%%%%%%%%%%%%%
%%%%%%%%%%%%%%%%%%%%%%%%%%%%%%%%%%%%%%%%%%%%%%%%%%%%%
%% 01. CVICENI
%%%%%%%%%%%%%%%%%%%%%%%%%%%%%%%%%%%%%%%%%%%%%%%%%%%%%
%%%%%%%%%%%%%%%%%%%%%%%%%%%%%%%%%%%%%%%%%%%%%%%%%%%%%
\clearpage
\newpage
\section{Cvičení 1: Vektory a matice}
Vektory a matice, operace s vektory, lineární kombinace vektorů, lineární závislost a nezávislost, operace s maticemi, hodnost matice (i s parametrem). Soustavy lin. algebraických rovnic. Frobeniova věta.

\begin{boxx}
This text is formatted within the \texttt{boxed} environment.
\end{boxx}

\begin{colbox}{DDDDDD}
\textbf{Příklad: }\\ \textit{some text here}
\end{colbox}

%%%%%%%%%%%%%%%%%%%%%%%%%%%%%%%%%%%%%%%%%%%%%%%%%%%%%
%%%%%%%%%%%%%%%%%%%%%%%%%%%%%%%%%%%%%%%%%%%%%%%%%%%%%
%% 02. CVICENI
%%%%%%%%%%%%%%%%%%%%%%%%%%%%%%%%%%%%%%%%%%%%%%%%%%%%%
%%%%%%%%%%%%%%%%%%%%%%%%%%%%%%%%%%%%%%%%%%%%%%%%%%%%%
\clearpage
\newpage
\section{Cviceni 2}

%%%%%%%%%%%%%%%%%%%%%%%%%%%%%%%%%%%%%%%%%%%%%%%%%%%%%
%%%%%%%%%%%%%%%%%%%%%%%%%%%%%%%%%%%%%%%%%%%%%%%%%%%%%
%% 03. CVICENI
%%%%%%%%%%%%%%%%%%%%%%%%%%%%%%%%%%%%%%%%%%%%%%%%%%%%%
%%%%%%%%%%%%%%%%%%%%%%%%%%%%%%%%%%%%%%%%%%%%%%%%%%%%%
\clearpage
\newpage
\section{Cviceni 3}

%%%%%%%%%%%%%%%%%%%%%%%%%%%%%%%%%%%%%%%%%%%%%%%%%%%%%
%%%%%%%%%%%%%%%%%%%%%%%%%%%%%%%%%%%%%%%%%%%%%%%%%%%%%
%% 04. CVICENI
%%%%%%%%%%%%%%%%%%%%%%%%%%%%%%%%%%%%%%%%%%%%%%%%%%%%%
%%%%%%%%%%%%%%%%%%%%%%%%%%%%%%%%%%%%%%%%%%%%%%%%%%%%%
\clearpage
\newpage
\section{Cviceni 4}

%%%%%%%%%%%%%%%%%%%%%%%%%%%%%%%%%%%%%%%%%%%%%%%%%%%%%
%%%%%%%%%%%%%%%%%%%%%%%%%%%%%%%%%%%%%%%%%%%%%%%%%%%%%
%% 05. CVICENI
%%%%%%%%%%%%%%%%%%%%%%%%%%%%%%%%%%%%%%%%%%%%%%%%%%%%%
%%%%%%%%%%%%%%%%%%%%%%%%%%%%%%%%%%%%%%%%%%%%%%%%%%%%%
\clearpage
\newpage
\section{Cviceni 5}

%%%%%%%%%%%%%%%%%%%%%%%%%%%%%%%%%%%%%%%%%%%%%%%%%%%%%
%%%%%%%%%%%%%%%%%%%%%%%%%%%%%%%%%%%%%%%%%%%%%%%%%%%%%
%% 06. CVICENI
%%%%%%%%%%%%%%%%%%%%%%%%%%%%%%%%%%%%%%%%%%%%%%%%%%%%%
%%%%%%%%%%%%%%%%%%%%%%%%%%%%%%%%%%%%%%%%%%%%%%%%%%%%%
\clearpage
\newpage
\section{Cviceni 6}

%%%%%%%%%%%%%%%%%%%%%%%%%%%%%%%%%%%%%%%%%%%%%%%%%%%%%
%%%%%%%%%%%%%%%%%%%%%%%%%%%%%%%%%%%%%%%%%%%%%%%%%%%%%
%% 07. CVICENI
%%%%%%%%%%%%%%%%%%%%%%%%%%%%%%%%%%%%%%%%%%%%%%%%%%%%%
%%%%%%%%%%%%%%%%%%%%%%%%%%%%%%%%%%%%%%%%%%%%%%%%%%%%%
\clearpage
\newpage
\section{Cviceni 7}

%%%%%%%%%%%%%%%%%%%%%%%%%%%%%%%%%%%%%%%%%%%%%%%%%%%%%
%%%%%%%%%%%%%%%%%%%%%%%%%%%%%%%%%%%%%%%%%%%%%%%%%%%%%
%% 08. CVICENI
%%%%%%%%%%%%%%%%%%%%%%%%%%%%%%%%%%%%%%%%%%%%%%%%%%%%%
%%%%%%%%%%%%%%%%%%%%%%%%%%%%%%%%%%%%%%%%%%%%%%%%%%%%%
\clearpage
\newpage
\section{Cvičení 8: Lokální extrémy funkcí}
%Lokální extrémy funkce 2 proměnných (stacionární body, Hessova matice). Metoda nejmenších čtverců.

%%%%%%%%%%%%%%%%%%%%%%%%%%
%%%%% TEORIE
%%%%%%%%%%%%%%%%%%%%%%%%%%
\noindent\fbox{%
    \parbox{\textwidth}{%
        \framebox[1.0\columnwidth]{Extremálnost bodu}\nonumber
        \vspace{2mm}
        Pokud pro danou funkci $f(x,y)$ platí v bodě $(x_0, y_0)$ následující podmínka
        \begin{align}
            \frac{\partial f}{\partial x}(x_0, y_0)=0 \quad \land \quad \frac{\partial f}{\partial y}(x_0, y_0)=0,
        \end{align}
        potom bod $(x_0, y_0)$ nazýváme \textbf{stacionárním bodem} funkce $f(x,y)$. Další rozhodování o extremálnosti leží ve vyčíslení Hessiánu $H_f(x,y)$, který je definovám jako determinant Hessovy matice
        \begin{align}
            \begin{pmatrix}
                \frac{\partial^2 f}{\partial x^2}(x, y) & \frac{\partial^2 f}{\partial x\partial y}(x, y) \\
                \frac{\partial^2 f}{\partial y\partial x}(x, y) &
                \frac{\partial^2 f}{\partial y^2}(x, y).
            \end{pmatrix}
        \end{align}
        Tedy Hessián lze vyjádřit jako
        \begin{align}
            H_f(x,y)=\frac{\partial^2 f}{\partial x^2}\frac{\partial^2 f}{\partial y^2}-\left(\frac{\partial^2 f}{\partial x\partial y}\right)^2.
        \end{align}
        Potom platí:
        \begin{enumerate}
            \item Je-li $H_f(x_0, y_0)>0$ má funkce $f$ v bodě $(x_0, y_0)$\ potom
                \begin{enumerate}
                    \item $\frac{\partial^2 f}{\partial x^2}(x_0, y_0)>0$ má funkce v bodě $(x_0, y_0)$ ostré lokální minimum,
                    \item $\frac{\partial^2 f}{\partial x^2}(x_0, y_0)<0$ má funkce v bodě $(x_0, y_0)$ ostré lokální maximum.
                \end{enumerate}
            \item Je-li $H_f(x_0, y_0)<0$ má funkce $f$ v bodě $(x_0, y_0)$ sedlový bod
            \item Je-li $H_f(x_0, y_0)=0$ nevíme zda má $f$ v bodě $(x_0, y_0)$ lokální extrém
        \end{enumerate}
    }%
}

%%%%%%%%%%%%%%%%%%%%%%%%%%
%%%%% PŘÍKLAD
%%%%%%%%%%%%%%%%%%%%%%%%%%
\begin{colbox}{DDDDDD}
\begin{exmp}
Určete lokální extrémy funkce $f(x,y)=xy\e{x-y^2/2}$.
\end{exmp}
\end{colbox}
Podezřelé body: $(x_0, y_0)=(0, 0); (x_1, y_1)=(-1, 1); (x_2, y_2)=(-1, -1)$
\begin{center}
%\centering
\begin{tabular}{cccc}
    \centering
    Podezřelý bod & Hessova matice & Hessián & Klasifikace \\
    \hline
    Bod $(0, 0)$ & $\bar{H}=
    \begin{pmatrix}
        0 & 1 \\
        1 & 0
    \end{pmatrix} $ & $H_f=-1$ & Sedlo \\
    Bod $(-1, 1)$ & $\bar{H}=
    \begin{pmatrix}
        1 & 0 \\
        0 & 2
    \end{pmatrix} $ & $H_f=2$ & Lokální minimum \\
    Bod $(-1, -1)$ & $\bar{H}=
    \begin{pmatrix}
        -1 & 0 \\
        0 & -2
    \end{pmatrix} $ & $H_f=2$ & Lokální maximum \\
\end{tabular}
\end{center}

%%%%%%%%%%%%%%%%%%%%%%%%%%
%%%%% PŘÍKLAD
%%%%%%%%%%%%%%%%%%%%%%%%%%
\begin{colbox}{DDDDDD}
\begin{exmp}
Určete lokální extrémy funkce $f(x,y)=xy\e{x-y^2/2}$.
\end{exmp}
\end{colbox}

%%%%%%%%%%%%%%%%%%%%%%%%%%
%%%%% PŘÍKLAD
%%%%%%%%%%%%%%%%%%%%%%%%%%
\begin{colbox}{ADD8E6}
\textbf{Cvičení:}
\end{colbox}

%%%%%%%%%%%%%%%%%%%%%%%%%%
%%%%% TEORIE
%%%%%%%%%%%%%%%%%%%%%%%%%%
\noindent\fbox{%
    \parbox{\textwidth}{%
        \framebox[1.0\columnwidth]{Metoda nejmenších čtverců}\nonumber
        \vspace{2mm}
        Pokud dvojicí získaných dat $\{x_i, y_i\}_{i=1}^n$ chci proložit co nejlépe lineární křivku $y=ax+b$, potom lze parametry $a, b$ najít optimalizační úlohou. Uvažujme nejprve, že celková chyba $e(a,b)$ je dána jako čtverec rozdílu mezi změřenými a odhadovanými daty, tedy
        \begin{align*}
            e(a,b)=\sum_{i=1}^n\left(ax_i + b - y_i\right)^2,
        \end{align*}
        potom hledám nejlepší parametry $a^*, b^*$ takové, že minimalizují celkovou chybu. Tedy pro tyto parametry musí platit:
        \begin{align*}
            \frac{\partial e}{\partial a}(a^*, b^*)=0,\quad\land\quad \frac{\partial e}{\partial b}(a^*, b^*)=0,
        \end{align*}
        Tedy
        \begin{align*}
            2\sum_{i=1}^n\left(ax_i + b - y_i\right)x_i=0, \\
            2\sum_{i=1}^n\left(ax_i + b - y_i\right)=0.
        \end{align*}
        A po úpravě konečně dostáváme soustavu
        \begin{align*}
            a\sum_{i=1}^n x_i^2 + b\sum_{i=1}^n x_i&=\sum_{i=1}^n x_iy_i, \\
            a\sum_{i=1}^n x_i + bn&=\sum_{i=1}^n y_i\\
        \end{align*}
        
    }%
}

%%%%%%%%%%%%%%%%%%%%%%%%%%
%%%%% PŘÍKLAD
%%%%%%%%%%%%%%%%%%%%%%%%%%
\begin{colbox}{DDDDDD}
\begin{exmp}
Metodou nejmenších čtverců aproximujte data křivkou $y=ax+b$.

\begin{center}
\begin{tabular}{|c|c|c|c|c|c|}
    \hline
    $x_i$ & 0 & 1 & 3 & 5 & 6 \\
    \hline
    $y_i$ & 5 & 3 & 3 & 2 & 1 \\
    \hline
\end{tabular}
\end{center}

\end{exmp}
\end{colbox}

Řešení: $y=-0.538x + 4.415$

%%%%%%%%%%%%%%%%%%%%%%%%%%%%%%%%%%%%%%%%%%%%%%%%%%%%%
%%%%%%%%%%%%%%%%%%%%%%%%%%%%%%%%%%%%%%%%%%%%%%%%%%%%%
%% 09. CVICENI
%%%%%%%%%%%%%%%%%%%%%%%%%%%%%%%%%%%%%%%%%%%%%%%%%%%%%
%%%%%%%%%%%%%%%%%%%%%%%%%%%%%%%%%%%%%%%%%%%%%%%%%%%%%
\clearpage
\newpage
\section{Cvičení 9: Implicitně definované funkce}

%%%%%%%%%%%%%%%%%%%%%%%%%%
%%%%% TEORIE
%%%%%%%%%%%%%%%%%%%%%%%%%%
\noindent\fbox{%
    \parbox{\textwidth}{%
        \framebox[1.0\columnwidth]{Existence implicitně zadané funkce}\nonumber
        \vspace{2mm}
        Pokud platí, že funkce $y=f(x)$ je zadána implicitně jako $F(x,y)=0$ na nějakém okolí bodu $(x_0, y_0)$, potom v tomto bodu musí platit
        \begin{align}
            F(x_0, y_0)=0 \quad \land \quad \frac{\partial F}{\partial y}(x_0, y_0)\neq 0
        \end{align}
        Jsou-li naplněny všechny požadavky na funkci a její spojitost, pak je derivace funkce $f$ definována vztahem
        \begin{align}
            f'(x)=-\frac{\frac{\partial F}{\partial x}(x, f(x))}{\frac{\partial F}{\partial y}(x, f(x))}
        \end{align}
        Podobně platí, že v okolí bodu $(x_0, y_0, z_0)$ rovnice $F(x,y,z)=0$ definuje implicitně zadanou funkci dvou proměnných $z=f(x,y)$ pokud
        \begin{align}
            F(x_0, y_0, z_0)=0 \quad \land \quad \frac{\partial F}{\partial z}(x_0, y_0, z_0)\neq 0
        \end{align}
        Obdobně jsou také definovány derivace funkce $f$ jako
        \begin{align}
            \frac{\partial f}{\partial x}(x,y)=-\frac{\frac{\partial F}{\partial x}(x, y, f(x, y))}{\frac{\partial F}{\partial z}(x, y f(x, y))}, \quad
            \frac{\partial f}{\partial y}(x,y)=-\frac{\frac{\partial F}{\partial y}(x, y, f(x, y))}{\frac{\partial F}{\partial z}(x, y f(x, y))}.
        \end{align}
        Pro výpočet derivací je často výhodnější použít jiný postup. Vezmeme rovnici $F(x, y)=0$ a dosadíme $y=f(x)$, čímž získáme funkci jedné proměnné $F(x, f(x))=0$. Tu ale nyní derivovat podle $x$. 
    }%
}

%%%%%%%%%%%%%%%%%%%%%%%%%%
%%%%% PŘÍKLAD
%%%%%%%%%%%%%%%%%%%%%%%%%%
\begin{colbox}{DDDDDD}
\begin{exmp}
Implicitně zadaná funkce $y^3 + 1 = xy$ v okolí bodu $(2, 1)$.
\end{exmp}
\end{colbox}
Řešení: $y'=\frac{y}{3y^2-x}, y''=\frac{-2xy}{(3y^2-x)^3}$

%%%%%%%%%%%%%%%%%%%%%%%%%%
%%%%% PŘÍKLAD
%%%%%%%%%%%%%%%%%%%%%%%%%%
\begin{colbox}{DDDDDD}
\begin{exmp}
Implicitně definovaná funkce $\text{cos}\left(4x-y\right)=\frac{\sqrt{2}}{2}$ v okolí bodu $\left(\frac{\pi}{12}, \frac{\pi}{12}\right)$.
\end{exmp}
\end{colbox}
%Řešení: $y'=\frac{y}{3y^2-x}, y''=\frac{-2xy}{(3y^2-x)^3}$
%Implicitně zadané funkce (1 a 2 proměnných), derivace implicitně zadané funkce.

%%%%%%%%%%%%%%%%%%%%%%%%%%%%%%%%%%%%%%%%%%%%%%%%%%%%%
%%%%%%%%%%%%%%%%%%%%%%%%%%%%%%%%%%%%%%%%%%%%%%%%%%%%%
%% 10. CVICENI
%%%%%%%%%%%%%%%%%%%%%%%%%%%%%%%%%%%%%%%%%%%%%%%%%%%%%
%%%%%%%%%%%%%%%%%%%%%%%%%%%%%%%%%%%%%%%%%%%%%%%%%%%%%
\clearpage
\newpage
\section{Cvičení 10: Parametrické křivky a křivkový integrál}

%%%%%%%%%%%%%%%%%%%%%%%%%%
%%%%% PŘÍKLAD
%%%%%%%%%%%%%%%%%%%%%%%%%%
\begin{colbox}{DDDDDD}
\begin{exmp}
    Křivka $\mathcal{K}$ je grafem funkce $y=\e{-x}, x\in \langle 0, 3\rangle$ a je orientovaná souhlasně s klesajícím $x$. Křivku $\mathcal{K}$ nakreslete včetně orientace a napište nějakou její parametrizaci.
\end{exmp}
\end{colbox}

Hledaná parametrizace je
\begin{align*}
    \vect{r}(t) = (-t, \e{t}),\quad t\in \langle -3, 0\rangle 
\end{align*}

%%%%%%%%%%%%%%%%%%%%%%%%%%
%%%%% PŘÍKLAD
%%%%%%%%%%%%%%%%%%%%%%%%%%
\begin{colbox}{DDDDDD}
\begin{exmp}
    Napište parametrické rovnice kružnice zadané obecnou rovnici 
    \begin{align*}
        x^2 + y^2 -6x - 8y = 0.
    \end{align*}
    Ověřte, zda kružnice prochází počátkem kartézské soustavy souřadnic.
\end{exmp}
\end{colbox}

Úpravou dostáváme středový tvar
\begin{align*}
    (x-3)^2 + (y-4)^2=25
\end{align*}
Parametrické rovnice jsou 
\begin{align*}
    x=5\text{cos}t + 3,\\
    y=5\text{sin}t + 4,\\
    t\in \langle 0, 2\pi\rangle
\end{align*}

%%%%%%%%%%%%%%%%%%%%%%%%%%
%%%%% PŘÍKLAD
%%%%%%%%%%%%%%%%%%%%%%%%%%
\begin{colbox}{DDDDDD}
\begin{exmp}
    Napište parametrické rovnice úsečky dané počátečním bodem $A=(1, \sqrt{2})$ a koncovým bodem $B=(\sqrt{8}, \sqrt{8})$.
\end{exmp}
\end{colbox}

Dostáváme 
\begin{align*}
    r(t)=A + (B-A)t, \\
    x(t)=1 + (\sqrt{8} - 1)t,\\
    y(t)=\sqrt{2} + \sqrt{2}t,\\
    t\in \langle 0, 1\pi\rangle
\end{align*}

%%%%%%%%%%%%%%%%%%%%%%%%%%
%%%%% TEORIE
%%%%%%%%%%%%%%%%%%%%%%%%%%
\noindent\fbox{%
    \parbox{\textwidth}{%
        \framebox[1.0\columnwidth]{Křivkový integrál}\nonumber
        \vspace{2mm}
        \begin{align*}
        \int_\mathcal{K}\vect{F}\,\mathrm{d}\vect{r}=\int_a^b\vect{F}(\vect{r}(t))\cdot\vect{r}'(t)\,\mathrm{d}t
        \end{align*}
    }%
}

%%%%%%%%%%%%%%%%%%%%%%%%%%
%%%%% PŘÍKLAD
%%%%%%%%%%%%%%%%%%%%%%%%%%
\begin{colbox}{DDDDDD}
\begin{exmp}
    Vypočítejte křivkový integrál z vektorové funkce $F(x,y,z)=(y^2-z^2, 2yz, -x^2)$ podél kladně orientované křivky s parametrizací $r(t)=(t, t^2, t^3), t\in\langle 0,1\rangle$.
\end{exmp}
\end{colbox}
Řešení: 
\begin{align*}
    \int_\mathcal{K}\vect{F}\,\mathrm{d}\vect{r}=\frac{1}{35}
\end{align*}

%%%%%%%%%%%%%%%%%%%%%%%%%%
%%%%% PŘÍKLAD
%%%%%%%%%%%%%%%%%%%%%%%%%%
\begin{colbox}{DDDDDD}
\begin{exmp}
    Vypočítejte křivkový integrál z vektorové funkce $F(x,y)=(y,x)$ podél čtvrtkružnice $x^2+y^2=a^2, x\geq 0, y\geq 0, a> 0$ s počátečním bodem $(a, 0)$ a koncovým bodem $(0,a)$.
\end{exmp}
\end{colbox}
Řešení: 
\begin{align*}
    \int_\mathcal{K}\vect{F}\,\mathrm{d}\vect{r}=0
\end{align*}

%%%%%%%%%%%%%%%%%%%%%%%%%%
%%%%% PŘÍKLAD
%%%%%%%%%%%%%%%%%%%%%%%%%%
\begin{colbox}{DDDDDD}
\begin{exmp}
    Vypočítejte křivkový integrál z vektorové funkce $F(x,y)=(x^2+y^2,x^2-y^2)$ podél křivky $y=1-|1-x|, x in \langle 0,2\rangle$ s počátečním bodem $(0, 0)$.
\end{exmp}
\end{colbox}
Řešení: 
\begin{align*}
    \int_\mathcal{K}\vect{F}\,\mathrm{d}\vect{r}=\frac{4}{3}
\end{align*}

%%%%%%%%%%%%%%%%%%%%%%%%%%%%%%%%%%%%%%%%%%%%%%%%%%%%%
%%%%%%%%%%%%%%%%%%%%%%%%%%%%%%%%%%%%%%%%%%%%%%%%%%%%%
%% 11. CVICENI
%%%%%%%%%%%%%%%%%%%%%%%%%%%%%%%%%%%%%%%%%%%%%%%%%%%%%
%%%%%%%%%%%%%%%%%%%%%%%%%%%%%%%%%%%%%%%%%%%%%%%%%%%%%
\clearpage
\newpage
\section{Cviceni 11}

Nez ávislost křivkov ého integr álu na integrační cestě. Potenciál vektorového pole. Určování potenciálu v R2 a v R3. Neučí se pojem jednoduše souvislá oblast, místo ní uvádíme slabší větu pro konvexní oblast.

%%%%%%%%%%%%%%%%%%%%%%%%%%%%%%%%%%%%%%%%%%%%%%%%%%%%%
%%%%%%%%%%%%%%%%%%%%%%%%%%%%%%%%%%%%%%%%%%%%%%%%%%%%%
%% 12. CVICENI
%%%%%%%%%%%%%%%%%%%%%%%%%%%%%%%%%%%%%%%%%%%%%%%%%%%%%
%%%%%%%%%%%%%%%%%%%%%%%%%%%%%%%%%%%%%%%%%%%%%%%%%%%%%
\clearpage
\newpage
\section{Cviceni 12}

%%%%%%%%%%%%%%%%%%%%%%%%%%%%%%%%%%%%%%%%%%%%%%%%%%%%%
%%%%%%%%%%%%%%%%%%%%%%%%%%%%%%%%%%%%%%%%%%%%%%%%%%%%%
%% 13. CVICENI
%%%%%%%%%%%%%%%%%%%%%%%%%%%%%%%%%%%%%%%%%%%%%%%%%%%%%
%%%%%%%%%%%%%%%%%%%%%%%%%%%%%%%%%%%%%%%%%%%%%%%%%%%%%
\clearpage
\newpage
\section{Cviceni 13}


\end{document}