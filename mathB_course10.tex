\documentclass{article}

% Language setting
% Replace `english' with e.g. `spanish' to change the document language
\usepackage[english]{babel}

% Set page size and margins
% Replace `letterpaper' with `a4paper' for UK/EU standard size
\usepackage[letterpaper,top=2cm,bottom=2cm,left=3cm,right=3cm,marginparwidth=1.75cm]{geometry}

% Useful packages
\usepackage{amsmath}
\usepackage{graphicx}
\usepackage[colorlinks=true, allcolors=blue]{hyperref}
\usepackage{empheq}
\usepackage{amsthm}
\usepackage{bm}
\usepackage{color}
\usepackage{xcolor}

\theoremstyle{definition}
\newtheorem{exmp}{Příklad}[section]
\setcounter{section}{10}

\newcommand{\e}[1]{\mathrm{e}^{#1}}
\newcommand{\vect}[1]{\bm{#1}}

\newenvironment{boxx}
    {\begin{center}
    \begin{tabular}{|p{0.9\textwidth}|}
    \hline\\
    }
    { 
    \\\\\hline
    \end{tabular} 
    \end{center}
    }

\newsavebox{\selvestebox}
\newenvironment{colbox}[1]
  {\newcommand\colboxcolor{#1}%
   \begin{lrbox}{\selvestebox}%
   \begin{minipage}{\dimexpr\columnwidth-2\fboxsep\relax}}
  {\end{minipage}\end{lrbox}%
   \begin{center}
   \colorbox[HTML]{\colboxcolor}{\usebox{\selvestebox}}
   \end{center}}

\graphicspath{{figs/}}

\title{Matematika B - Cvičení 10}
\date{}
%\author{You}

\begin{document}
\maketitle

%%%%%%%%%%%%%%%%%%%%%%%%%%%%%%%%%%%%%%%%%%%%%%%%%%%%%
%%%%%%%%%%%%%%%%%%%%%%%%%%%%%%%%%%%%%%%%%%%%%%%%%%%%%
%% 10. CVICENI
%%%%%%%%%%%%%%%%%%%%%%%%%%%%%%%%%%%%%%%%%%%%%%%%%%%%%
%%%%%%%%%%%%%%%%%%%%%%%%%%%%%%%%%%%%%%%%%%%%%%%%%%%%%
%\clearpage
%\newpage
%\section{Cvičení 10: Parametrické křivky a křivkový integrál}

%%%%%%%%%%%%%%%%%%%%%%%%%%
%%%%% PŘÍKLAD
%%%%%%%%%%%%%%%%%%%%%%%%%%
\begin{colbox}{DDDDDD}
\begin{exmp}
    Křivka $\mathcal{K}$ je grafem funkce $y=\e{-x}, x\in \langle 0, 3\rangle$ a je orientovaná souhlasně s klesajícím $x$. Křivku $\mathcal{K}$ nakreslete včetně orientace a napište nějakou její parametrizaci.
\end{exmp}
\end{colbox}

Hledaná parametrizace je
\begin{align*}
    \vect{r}(t) = (-t, \e{t}),\quad t\in \langle -3, 0\rangle 
\end{align*}

%%%%%%%%%%%%%%%%%%%%%%%%%%
%%%%% PŘÍKLAD
%%%%%%%%%%%%%%%%%%%%%%%%%%
\begin{colbox}{DDDDDD}
\begin{exmp}
    Napište parametrické rovnice kružnice zadané obecnou rovnici 
    \begin{align*}
        x^2 + y^2 -6x - 8y = 0.
    \end{align*}
    Ověřte, zda kružnice prochází počátkem kartézské soustavy souřadnic.
\end{exmp}
\end{colbox}

Úpravou dostáváme středový tvar
\begin{align*}
    (x-3)^2 + (y-4)^2=25
\end{align*}
Parametrické rovnice jsou 
\begin{align*}
    x=5\text{cos}t + 3,\\
    y=5\text{sin}t + 4,\\
    t\in \langle 0, 2\pi\rangle
\end{align*}

%%%%%%%%%%%%%%%%%%%%%%%%%%
%%%%% PŘÍKLAD
%%%%%%%%%%%%%%%%%%%%%%%%%%
\begin{colbox}{DDDDDD}
\begin{exmp}
    Napište parametrické rovnice úsečky dané počátečním bodem $A=(1, \sqrt{2})$ a koncovým bodem $B=(\sqrt{8}, \sqrt{8})$.
\end{exmp}
\end{colbox}

Dostáváme 
\begin{align*}
    r(t)=A + (B-A)t, \\
    x(t)=1 + (\sqrt{8} - 1)t,\\
    y(t)=\sqrt{2} + \sqrt{2}t,\\
    t\in \langle 0, 1\pi\rangle
\end{align*}

%%%%%%%%%%%%%%%%%%%%%%%%%%
%%%%% TEORIE
%%%%%%%%%%%%%%%%%%%%%%%%%%
\noindent\fbox{%
    \parbox{\textwidth}{%
        \framebox[1.0\columnwidth]{Křivkový integrál}\nonumber
        \vspace{2mm}
        \begin{align*}
        \int_\mathcal{K}\vect{F}\,\mathrm{d}\vect{r}=\int_a^b\vect{F}(\vect{r}(t))\cdot\vect{r}'(t)\,\mathrm{d}t
        \end{align*}
    }%
}

%%%%%%%%%%%%%%%%%%%%%%%%%%
%%%%% PŘÍKLAD
%%%%%%%%%%%%%%%%%%%%%%%%%%
\begin{colbox}{DDDDDD}
\begin{exmp}
    Vypočítejte křivkový integrál z vektorové funkce $F(x,y,z)=(y^2-z^2, 2yz, -x^2)$ podél kladně orientované křivky s parametrizací $r(t)=(t, t^2, t^3), t\in\langle 0,1\rangle$.
\end{exmp}
\end{colbox}
Řešení: 
\begin{align*}
    \int_\mathcal{K}\vect{F}\,\mathrm{d}\vect{r}=\frac{1}{35}
\end{align*}

%%%%%%%%%%%%%%%%%%%%%%%%%%
%%%%% PŘÍKLAD
%%%%%%%%%%%%%%%%%%%%%%%%%%
\begin{colbox}{DDDDDD}
\begin{exmp}
    Vypočítejte křivkový integrál z vektorové funkce $F(x,y)=(y,x)$ podél čtvrtkružnice $x^2+y^2=a^2, x\geq 0, y\geq 0, a> 0$ s počátečním bodem $(a, 0)$ a koncovým bodem $(0,a)$.
\end{exmp}
\end{colbox}
Řešení: 
\begin{align*}
    \int_\mathcal{K}\vect{F}\,\mathrm{d}\vect{r}=0
\end{align*}

%%%%%%%%%%%%%%%%%%%%%%%%%%
%%%%% PŘÍKLAD
%%%%%%%%%%%%%%%%%%%%%%%%%%
\begin{colbox}{DDDDDD}
\begin{exmp}
    Vypočítejte křivkový integrál z vektorové funkce $F(x,y)=(x^2+y^2,x^2-y^2)$ podél křivky $y=1-|1-x|, x in \langle 0,2\rangle$ s počátečním bodem $(0, 0)$.
\end{exmp}
\end{colbox}
Řešení: 
\begin{align*}
    \int_\mathcal{K}\vect{F}\,\mathrm{d}\vect{r}=\frac{4}{3}
\end{align*}

\end{document}