\documentclass{article}

% Language setting
% Replace `english' with e.g. `spanish' to change the document language
\usepackage[english]{babel}

% Set page size and margins
% Replace `letterpaper' with `a4paper' for UK/EU standard size
\usepackage[letterpaper,top=2cm,bottom=2cm,left=3cm,right=3cm,marginparwidth=1.75cm]{geometry}

% Useful packages
\usepackage{amsmath}
\usepackage{graphicx}
\usepackage[colorlinks=true, allcolors=blue]{hyperref}
\usepackage{empheq}
\usepackage{amsthm}
\usepackage{bm}
\usepackage{color}
\usepackage{xcolor}

\theoremstyle{definition}
\newtheorem{exmp}{Příklad}[section]
\setcounter{section}{9}

\newcommand{\e}[1]{\mathrm{e}^{#1}}
\newcommand{\vect}[1]{\bm{#1}}

\newenvironment{boxx}
    {\begin{center}
    \begin{tabular}{|p{0.9\textwidth}|}
    \hline\\
    }
    { 
    \\\\\hline
    \end{tabular} 
    \end{center}
    }

\newsavebox{\selvestebox}
\newenvironment{colbox}[1]
  {\newcommand\colboxcolor{#1}%
   \begin{lrbox}{\selvestebox}%
   \begin{minipage}{\dimexpr\columnwidth-2\fboxsep\relax}}
  {\end{minipage}\end{lrbox}%
   \begin{center}
   \colorbox[HTML]{\colboxcolor}{\usebox{\selvestebox}}
   \end{center}}

\graphicspath{{figs/}}

\title{Matematika B - Cvičení 9}
\date{}
%\author{You}

\begin{document}
\maketitle

%%%%%%%%%%%%%%%%%%%%%%%%%%%%%%%%%%%%%%%%%%%%%%%%%%%%%
%%%%%%%%%%%%%%%%%%%%%%%%%%%%%%%%%%%%%%%%%%%%%%%%%%%%%
%% 09. CVICENI
%%%%%%%%%%%%%%%%%%%%%%%%%%%%%%%%%%%%%%%%%%%%%%%%%%%%%
%%%%%%%%%%%%%%%%%%%%%%%%%%%%%%%%%%%%%%%%%%%%%%%%%%%%%
%\clearpage
%\newpage
%\section{Cvičení 9: Implicitně definované funkce}

%%%%%%%%%%%%%%%%%%%%%%%%%%
%%%%% TEORIE
%%%%%%%%%%%%%%%%%%%%%%%%%%
\noindent\fbox{%
    \parbox{\textwidth}{%
        \framebox[1.0\columnwidth]{Existence implicitně zadané funkce}\nonumber
        \vspace{2mm}
        Pokud platí, že funkce $y=f(x)$ je zadána implicitně jako $F(x,y)=0$ na nějakém okolí bodu $(x_0, y_0)$, potom v tomto bodu musí platit
        \begin{align}
            F(x_0, y_0)=0 \quad \land \quad \frac{\partial F}{\partial y}(x_0, y_0)\neq 0
        \end{align}
        Jsou-li naplněny všechny požadavky na funkci a její spojitost, pak je derivace funkce $f$ definována vztahem
        \begin{align}
            f'(x)=-\frac{\frac{\partial F}{\partial x}(x, f(x))}{\frac{\partial F}{\partial y}(x, f(x))}
        \end{align}
        Podobně platí, že v okolí bodu $(x_0, y_0, z_0)$ rovnice $F(x,y,z)=0$ definuje implicitně zadanou funkci dvou proměnných $z=f(x,y)$ pokud
        \begin{align}
            F(x_0, y_0, z_0)=0 \quad \land \quad \frac{\partial F}{\partial z}(x_0, y_0, z_0)\neq 0
        \end{align}
        Obdobně jsou také definovány derivace funkce $f$ jako
        \begin{align}
            \frac{\partial f}{\partial x}(x,y)=-\frac{\frac{\partial F}{\partial x}(x, y, f(x, y))}{\frac{\partial F}{\partial z}(x, y f(x, y))}, \quad
            \frac{\partial f}{\partial y}(x,y)=-\frac{\frac{\partial F}{\partial y}(x, y, f(x, y))}{\frac{\partial F}{\partial z}(x, y f(x, y))}.
        \end{align}
        Pro výpočet derivací je často výhodnější použít jiný postup. Vezmeme rovnici $F(x, y)=0$ a dosadíme $y=f(x)$, čímž získáme funkci jedné proměnné $F(x, f(x))=0$. Tu ale nyní derivovat podle $x$. 
    }%
}

%%%%%%%%%%%%%%%%%%%%%%%%%%
%%%%% PŘÍKLAD
%%%%%%%%%%%%%%%%%%%%%%%%%%
\begin{colbox}{DDDDDD}
\begin{exmp}
Implicitně zadaná funkce $y^3 + 1 = xy$ v okolí bodu $(2, 1)$.
\end{exmp}
\end{colbox}
Řešení: $y'=\frac{y}{3y^2-x}, y''=\frac{-2xy}{(3y^2-x)^3}$

%%%%%%%%%%%%%%%%%%%%%%%%%%
%%%%% PŘÍKLAD
%%%%%%%%%%%%%%%%%%%%%%%%%%
\begin{colbox}{DDDDDD}
\begin{exmp}
Implicitně definovaná funkce $\text{cos}\left(4x-y\right)=\frac{\sqrt{2}}{2}$ v okolí bodu $\left(\frac{\pi}{12}, \frac{\pi}{12}\right)$.
\end{exmp}
\end{colbox}
%Řešení: $y'=\frac{y}{3y^2-x}, y''=\frac{-2xy}{(3y^2-x)^3}$
%Implicitně zadané funkce (1 a 2 proměnných), derivace implicitně zadané funkce.

\end{document}